\documentclass[10pt]{article}
\input{style/coursHeadings}
\input{style/programHeadings}
\input{style/macros_SII}
\input{style/macros_Titres}
\input{style/macros_Frames}

%Si le boolen xp est vrai : compilation pour xabi
%Sinon compilation Damien
\newboolean{xp}
\setboolean{xp}{true}

\newboolean{prof}
\setboolean{prof}{false}

\usepackage[%
    pdftitle={CI 06 : Stat - Modélisation des AM},
    pdfauthor={Xavier Pessoles},
    colorlinks=true,
    linkcolor=blue,
    citecolor=magenta]{hyperref}


\def\discipline{Sciences Industrielles de l'Ingénieur}
\def\xxtitre{\ifthenelse{\boolean{xp}}{
CI 06 : Étude du comportement statique des systèmes}{}}

\def\xxsoustitre{\ifthenelse{\boolean{xp}}{
Chapitre 1 -- Modélisation des Actions Mécaniques}{
Partie  -- }}

\def\xxauteur{\ifthenelse{\boolean{xp}}{
Stéphane Genouël
%Xavier \textsc{Pessoles}
}{}}

\def\xxpied{\ifthenelse{\boolean{xp}}{
CI 06 : Statique\\
Ch. 1 : Modélisation des AM -- TD 2 -- Lois de Coulomb}{
\xxtitre}}

\def\xxcathegorie{\ifthenelse{\boolean{xp}}{
2013 -- 2014 \\
Xavier \textsc{Pessoles}}{}}





%---------------------------------------------------------------------------


\begin{document}

\ifthenelse{\boolean{xp}}{\input{style/enteteXP}}{\input{style/enteteDI}}

\begin{center}
\Large{\textsc{Interrogation B}}
\end{center}

\subsection*{Cours}
\subparagraph*{}
\textit{Énoncer les lois de Coulomb.}



\subsection*{Exercice 2 : Embrayage à friction monodisque de véhicules automobiles (surfaces de friction plane)}
\setcounter{subparagraph}{0} 

\begin{minipage}[c]{.55\linewidth}
On modélise l'embrayage par 2 disques creux identiques (1 et 2) en contact grâce à une 
action axiale $\vect{Fa}$. 
 
Le rayon intérieur des 2 disques vaut : $R_{min}$. 
Le rayon extérieur des 2 disques vaut : $R_{max}$. 
On donne $f$ le facteur d’adhérence entre les deux pièces. 



\end{minipage}\hfill
\begin{minipage}[c]{.4\linewidth}
\begin{center}
\includegraphics[width=.9\textwidth]{images/fig_07}
\end{center}
\end{minipage}

\subparagraph{}
\textit{Refaire en grand les 2 schémas ci-dessus : un dans le plan $(\vect{y}, \vect{z})$ et l’autre dans le plan $(\vect{x}, \vect{y})$, en plaçant les actions élémentaires normale et tangentielle de 2 sur 1 en un point $Q$ quelconque de la surface de contact.}

\subparagraph{}
\textit{Exprimer $\vect{dF_{2\rightarrow 1}(Q)}$.}

\subparagraph{}
\textit{Déterminer le couple maximal transmissible en fonction de $p$ et des caractéristiques 
géométriques de l’embrayage.}

\end{document}

\subparagraph{}
\textit{}

