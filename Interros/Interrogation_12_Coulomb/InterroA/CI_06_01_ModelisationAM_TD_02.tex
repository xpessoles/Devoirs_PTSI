\documentclass[10pt]{article}
\input{style/coursHeadings}
\input{style/programHeadings}
\input{style/macros_SII}
\input{style/macros_Titres}
\input{style/macros_Frames}

%Si le boolen xp est vrai : compilation pour xabi
%Sinon compilation Damien
\newboolean{xp}
\setboolean{xp}{true}

\newboolean{prof}
\setboolean{prof}{false}

\usepackage[%
    pdftitle={CI 06 : Stat - Modélisation des AM},
    pdfauthor={Xavier Pessoles},
    colorlinks=true,
    linkcolor=blue,
    citecolor=magenta]{hyperref}


\def\discipline{Sciences Industrielles de l'Ingénieur}
\def\xxtitre{\ifthenelse{\boolean{xp}}{
CI 06 : Étude du comportement statique des systèmes}{}}

\def\xxsoustitre{\ifthenelse{\boolean{xp}}{
Chapitre 1 -- Modélisation des Actions Mécaniques}{
Partie  -- }}

\def\xxauteur{\ifthenelse{\boolean{xp}}{
Stéphane Genouël
%Xavier \textsc{Pessoles}
}{}}

\def\xxpied{\ifthenelse{\boolean{xp}}{
CI 06 : Statique\\
Ch. 1 : Modélisation des AM -- TD 2 -- Lois de Coulomb}{
\xxtitre}}

\def\xxcathegorie{\ifthenelse{\boolean{xp}}{
2013 -- 2014 \\
Xavier \textsc{Pessoles}}{}}





%---------------------------------------------------------------------------


\begin{document}

\ifthenelse{\boolean{xp}}{\input{style/enteteXP}}{\input{style/enteteDI}}

\begin{center}
\Large{\textsc{Interrogation A}}
\end{center}

\subsection*{Cours}
\subparagraph*{}
\textit{Énoncer les lois de Coulomb.}



\subsection*{Exercice 1 : Assemblage par frettage}
\ifthenelse{\boolean{prof}}{}{}
\begin{minipage}[c]{.5\linewidth}
Le frettage consiste à encastrer deux pièces en utilisant le phénomène d’adhérence. 
 
Avant l’assemblage réalisé à l’aide d’une presse, l’arbre 1 
possède un diamètre légèrement supérieur à celui de l’alésage 
(trou cylindrique) de la pièce 2 dans laquelle il vient se loger. 
 
Après frettage, il subsiste donc une pression de contact $p$ 
(souvent supposée uniforme sur toute la surface de contact) 
entre les deux pièces. 

 
Les caractéristiques de cet assemblage par frettage sont les suivantes : 
\begin{itemize}
\item $R$ : rayon de l’arbre 1;
\item $L$ : longueur du contact; 
\item $f$ : facteur d’adhérence entre les deux pièces.
\end{itemize}


\end{minipage}\hfill
\begin{minipage}[c]{.48\linewidth}
\begin{center}
\includegraphics[width=\textwidth]{images/fig_01}
\end{center}
\end{minipage}


\subsection*{Couple maximal transmissible}

\begin{minipage}[c]{.55\linewidth}
Le couple (ou moment) maximal transmissible correspond à la valeur maximale 
de la composante sur l’axe $\vect{z}$ du moment résultant de l’action mécanique qui peut 
être transmise d’une pièce à l’autre sans qu’elles se désolidarisent. 
 
Pour simplifier notre étude, on considère la pièce 2 fixe et on cherche à 
déterminer la composante sur l’axe $\vect{z}$ du moment résultant de l’action mécanique 
à appliquer à la pièce 1 pour atteindre le glissement de 1/2 autour de $\vect{z}$.
 
\end{minipage}\hfill
\begin{minipage}[c]{.4\linewidth}
\begin{center}
\includegraphics[width=.9\textwidth]{images/fig_04}
\end{center}
\end{minipage}

\begin{minipage}[c]{.55\linewidth}
\subparagraph{}
\textit{Refaire en grand les 2 schémas ci-contre : un 
dans le plan $(\vect{y}, \vect{z})$ et l’autre dans le plan $(\vect{x}, \vect{y})$, 
en plaçant les actions élémentaires normale et 
tangentielle de 2 sur 1 en un point $Q$ 
quelconque de la surface de contact. }


\end{minipage}\hfill
\begin{minipage}[c]{.4\linewidth}
\begin{center}
\includegraphics[width=.9\textwidth]{images/fig_05}
\end{center}
\end{minipage}


\subparagraph{}
\textit{Exprimer $\vect{dF_{2\rightarrow 1}(Q)}$.}

\subparagraph{}
\textit{Déterminer le couple maximal transmissible en fonction de $p$ et des 
caractéristiques géométriques du frettage.}




\end{document}

\subparagraph{}
\textit{}

